\documentclass[12pt]{article}
\usepackage[a4paper,margin=0.75in]{geometry}
\usepackage[utf8]{inputenc}
\usepackage[OT1]{fontenc}
\usepackage[table,usenames,dvipsnames]{xcolor}
\usepackage{array}
\usepackage{varwidth}
\usepackage{tabularx}
\usepackage{amsmath}
\usepackage{float}
\usepackage{parskip}
\usepackage{hyperref}
\usepackage{forest}
\usepackage{enumitem}
\usepackage{graphicx}
\usepackage{tcolorbox}
\usepackage{forest}

\renewcommand*\familydefault{\sfdefault}

\definecolor{dkgreen}{rgb}{0,0.6,0}
\definecolor{gray}{rgb}{0.5,0.5,0.5}
\definecolor{mauve}{rgb}{0.58,0,0.82}

\newtcolorbox{mybox}[3][]
{
  colframe = #2!25,
  colback  = #2!10,
  coltitle = #2!20!black,  
  title    = {#3},
  #1,
}

\hypersetup{
    colorlinks=true,
    linkcolor=blue,
    filecolor=magenta,      
    urlcolor=cyan,
    pdftitle={Overleaf Example},
    pdfpagemode=FullScreen,
}

\title{\textbf{COL334 Assignment 4 Part B}}
\author{Aniruddha Deb \\ \texttt{2020CS10869}}
\date{November 2022}

\begin{document}

\maketitle

\section*{(i) Routing Table Convergence}

\begin{enumerate}[label=(\alph*)]
    \item The plots are as follows:
    \begin{center}
        \includegraphics[width=0.44\textwidth]{../Q2/q2a_convergence_100ms.pdf}
        \includegraphics[width=0.44\textwidth]{../Q2/q2a_convergence_1000ms.pdf}
        \includegraphics[width=0.44\textwidth]{../Q2/q2a_convergence_20000ms.pdf}
        \includegraphics[width=0.44\textwidth]{../Q2/q2a_convergence_80000ms.pdf}
    \end{center}

    \item We see that for small delays, the routers converge quickly, while for the 
    20 second delay, the routers converge only after a long time, and for the 
    80 second delay, they do not converge at all.

\end{enumerate}

\section*{(ii) Link severing simulation}
\begin{enumerate}[label=(\alph*)]
    \item At T=121 seconds, we have the following tables:
    \begin{center}
        Router 1

        \begin{tabular}{|c|c|c|l|c|c|c|c|}
          \hline
          Destination & Gateway & Genmask & Flags & Metric & Ref & Use & Iface \\
          \hline
          10.0.0.0 & 0.0.0.0 & 255.255.255.0 & U & 1 & - & - & 1 \\
          \hline
        \end{tabular}

        Router 2

        \begin{tabular}{|c|c|c|l|c|c|c|c|}
          \hline
          Destination & Gateway & Genmask & Flags & Metric & Ref & Use & Iface \\
          \hline
          10.0.4.0 & 10.0.2.2 & 255.255.255.0 & UGS & 2 & - & - & 2 \\
          10.0.3.0 & 10.0.2.2 & 255.255.255.0 & UGS & 2 & - & - & 2 \\
          10.0.0.0 & 10.0.2.2 & 255.255.255.0 & UGS & 3 & - & - & 2 \\
          10.0.2.0 & 0.0.0.0 & 255.255.255.0 & U & 1 & - & - & 2 \\
          \hline
        \end{tabular}

        Router 3

        \begin{tabular}{|c|c|c|l|c|c|c|c|}
          \hline
          Destination & Gateway & Genmask & Flags & Metric & Ref & Use & Iface \\
          \hline
          10.0.1.0 & 10.0.2.1 & 255.255.255.0 & UGS & 16 & - & - & 1 \\
          10.0.2.0 & 0.0.0.0 & 255.255.255.0 & U & 1 & - & - & 1 \\
          10.0.4.0 & 0.0.0.0 & 255.255.255.0 & U & 1 & - & - & 3 \\
          \hline
        \end{tabular}
    \end{center}

    At T=180 seconds, we have the following tables:

    \begin{center}
        Router 1

        \begin{tabular}{|c|c|c|l|c|c|c|c|}
            \hline
            Destination & Gateway & Genmask & Flags & Metric & Ref & Use & Iface \\
            \hline
            10.0.0.0 & 0.0.0.0 & 255.255.255.0 & U & 1 & - & - & 1 \\
            \hline
        \end{tabular}

        Router 2

        \begin{tabular}{|c|c|c|l|c|c|c|c|}
            \hline
            Destination & Gateway & Genmask & Flags & Metric & Ref & Use & Iface \\
            \hline
            10.0.4.0 & 10.0.2.2 & 255.255.255.0 & UGS & 2 & - & - & 2 \\
            10.0.2.0 & 0.0.0.0 & 255.255.255.0 & U & 1 & - & - & 2 \\
            \hline
        \end{tabular}

        Router 3

        \begin{tabular}{|c|c|c|l|c|c|c|c|}
            \hline
            Destination & Gateway & Genmask & Flags & Metric & Ref & Use & Iface \\
            \hline
            10.0.1.0 & 10.0.2.1 & 255.255.255.0 & UGS & 16 & - & - & 1 \\
            10.0.2.0 & 0.0.0.0 & 255.255.255.0 & U & 1 & - & - & 1 \\
            10.0.4.0 & 0.0.0.0 & 255.255.255.0 & U & 1 & - & - & 3 \\
            \hline
        \end{tabular}
    \end{center}

    \item The Reachability graph is as follows:
    \begin{center}
        \includegraphics[width=0.8\textwidth]{../Q2/q2b_reachable.pdf}
    \end{center}

    \item At T=51 seconds, the path packets take is src $\rightarrow$ R1 
        $\rightarrow$ R3 $\rightarrow$ dest. The same path is taken at T=119 s, as 
        the link is not yet severed. At T=120 s, however, the R1 $\rightarrow$ R3 link is severed
        and we see that the destination node is not reachable at T=121 seconds, 
        as there are no paths to it (the link from R1 to R2 and R1 to R3 both 
        have been severed). 

\end{enumerate}

\section*{(iii) Link Reappearing Simulation}
\begin{enumerate}[label=(\alph*)]
    \item At T=70 seconds, we have the following tables:
    \begin{center}
        Router 1

        \begin{tabular}{|c|c|c|l|c|c|c|c|}
            \hline
            Destination & Gateway & Genmask & Flags & Metric & Ref & Use & Iface \\
            \hline
            10.0.0.0 & 0.0.0.0 & 255.255.255.0 & U & 1 & - & - & 1 \\
            \hline
        \end{tabular}

        Router 2

        \begin{tabular}{|c|c|c|l|c|c|c|c|}
            \hline
            Destination & Gateway & Genmask & Flags & Metric & Ref & Use & Iface \\
            \hline
            10.0.4.0 & 10.0.2.2 & 255.255.255.0 & UGS & 2 & - & - & 2 \\
            10.0.2.0 & 0.0.0.0 & 255.255.255.0 & U & 1 & - & - & 2 \\
            \hline
        \end{tabular}

        Router 3

        \begin{tabular}{|c|c|c|l|c|c|c|c|}
            \hline
            Destination & Gateway & Genmask & Flags & Metric & Ref & Use & Iface \\
            \hline
            10.0.2.0 & 0.0.0.0 & 255.255.255.0 & U & 1 & - & - & 1 \\
            10.0.4.0 & 0.0.0.0 & 255.255.255.0 & U & 1 & - & - & 3 \\
            \hline
        \end{tabular}
    \end{center}

    At T=180 seconds, we have the following tables:
    \begin{center}
        Router 1

        \begin{tabular}{|c|c|c|l|c|c|c|c|}
            \hline
            Destination & Gateway & Genmask & Flags & Metric & Ref & Use & Iface \\
            \hline
            10.0.4.0 & 10.0.3.2 & 255.255.255.0 & UGS & 2 & - & - & 3 \\
            10.0.2.0 & 10.0.1.2 & 255.255.255.0 & UGS & 2 & - & - & 2 \\
            10.0.0.0 & 0.0.0.0 & 255.255.255.0 & U & 1 & - & - & 1 \\
            10.0.1.0 & 0.0.0.0 & 255.255.255.0 & U & 1 & - & - & 2 \\
            10.0.3.0 & 0.0.0.0 & 255.255.255.0 & U & 1 & - & - & 3 \\
            \hline
        \end{tabular}

        Router 2

        \begin{tabular}{|c|c|c|l|c|c|c|c|}
            \hline
            Destination & Gateway & Genmask & Flags & Metric & Ref & Use & Iface \\
            \hline
            10.0.4.0 & 10.0.2.2 & 255.255.255.0 & UGS & 2 & - & - & 2 \\
            10.0.3.0 & 10.0.1.1 & 255.255.255.0 & UGS & 2 & - & - & 1 \\
            10.0.0.0 & 10.0.1.1 & 255.255.255.0 & UGS & 2 & - & - & 1 \\
            10.0.1.0 & 10.0.2.2 & 255.255.255.0 & UGS & 3 & - & - & 2 \\
            10.0.2.0 & 0.0.0.0 & 255.255.255.0 & U & 1 & - & - & 2 \\
            10.0.1.0 & 0.0.0.0 & 255.255.255.0 & U & 1 & - & - & 1 \\
            \hline
        \end{tabular}

        Router 3

        \begin{tabular}{|c|c|c|l|c|c|c|c|}
            \hline
            Destination & Gateway & Genmask & Flags & Metric & Ref & Use & Iface \\
            \hline
            10.0.1.0 & 10.0.3.1 & 255.255.255.0 & UGS & 2 & - & - & 2 \\
            10.0.0.0 & 10.0.3.1 & 255.255.255.0 & UGS & 2 & - & - & 2 \\
            10.0.2.0 & 0.0.0.0 & 255.255.255.0 & U & 1 & - & - & 1 \\
            10.0.3.0 & 10.0.2.1 & 255.255.255.0 & UGS & 3 & - & - & 1 \\
            10.0.4.0 & 0.0.0.0 & 255.255.255.0 & U & 1 & - & - & 3 \\
            10.0.3.0 & 0.0.0.0 & 255.255.255.0 & U & 1 & - & - & 2 \\
            \hline
        \end{tabular}      
    \end{center}

    \item The Reachability graph is as follows:
    \begin{center}
        \includegraphics[width=0.8\textwidth]{../Q2/q2b_reachable_3.pdf}
    \end{center}

    \item The destination node is reachable after T=140 seconds, as the 
        links to it have been reestablished at T=120 seconds. The reachability 
        does not immediately change after the link has been reestablished, eg. at 
        T=121 seconds, as the routing tables take time to update. 
\end{enumerate}

\subsection*{Appendix}

All files are as specified in the submission instructions. The rip-simple-network file was 
modified to create the connection and the simulation. The script saves the routing 
tables to the respective files, and the plotting was done using a Jupyter notebook 
and matplotlib (which has been attached, both as a notebook and as a script). 
The data files generated after the run are in the data directory. Note that the 
reachability and convergence CSV files were generated by hand after going 
through the routing tables that were printed to file.

\end{document}
