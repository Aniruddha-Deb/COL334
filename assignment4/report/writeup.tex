\documentclass[12pt]{article}
\usepackage[a4paper,margin=0.75in]{geometry}
\usepackage[utf8]{inputenc}
\usepackage[OT1]{fontenc}
\usepackage[table,usenames,dvipsnames]{xcolor}
\usepackage{array}
\usepackage{varwidth}
\usepackage{tabularx}
\usepackage{amsmath}
\usepackage{float}
\usepackage{parskip}
\usepackage{hyperref}
\usepackage{forest}
\usepackage{enumitem}
\usepackage{graphicx}
\usepackage{tcolorbox}
\usepackage{forest}

\renewcommand*\familydefault{\sfdefault}

\definecolor{dkgreen}{rgb}{0,0.6,0}
\definecolor{gray}{rgb}{0.5,0.5,0.5}
\definecolor{mauve}{rgb}{0.58,0,0.82}

\newtcolorbox{mybox}[3][]
{
  colframe = #2!25,
  colback  = #2!10,
  coltitle = #2!20!black,  
  title    = {#3},
  #1,
}

\hypersetup{
    colorlinks=true,
    linkcolor=blue,
    filecolor=magenta,      
    urlcolor=cyan,
    pdftitle={Overleaf Example},
    pdfpagemode=FullScreen,
}

\title{\textbf{COL334 Assignment 4 Part A}}
\author{Aniruddha Deb \\ \texttt{2020CS10869}}
\date{November 2022}

\begin{document}

\maketitle

\begin{enumerate}
    \item The plots for each source, with the new and old protocol are as follows:

    \begin{enumerate}[label=(\roman*)]
        \item For Connection 1 (from Node 1 to Node 3, starting at 1 s and ending at 
            20 s)
        \begin{center}
            TcpNewReno
            \includegraphics[width=0.88\textwidth]{../Q1/cwnd_plot_c1_TcpNewReno.pdf}
            TcpNewRenoPlus
            \includegraphics[width=0.88\textwidth]{../Q1/cwnd_plot_c1_TcpNewRenoPlus.pdf}
        \end{center}
        \item For Connection 2 (from Node 1 to Node 3, starting at 5 s and ending at
            25 s)
        \begin{center}
            TcpNewReno
            \includegraphics[width=0.88\textwidth]{../Q1/cwnd_plot_c2_TcpNewReno.pdf}
            TcpNewRenoPlus
            \includegraphics[width=0.88\textwidth]{../Q1/cwnd_plot_c2_TcpNewRenoPlus.pdf}
        \end{center}
        \item For Connection 3 (from Node 2 to Node 3, starting at 15 s and ending at 
            30 s)
        \begin{center}
            TcpNewReno
            \includegraphics[width=0.88\textwidth]{../Q1/cwnd_plot_c3_TcpNewReno.pdf}
            TcpNewRenoPlus
            \includegraphics[width=0.88\textwidth]{../Q1/cwnd_plot_c3_TcpNewRenoPlus.pdf}
        \end{center}
    \end{enumerate}

    \item We see that TcpNewRenoPlus uses a linear increase in the congestion control 
        phase and reaches very high congestion window values, whereas TcpNewReno 
        does not. Also, TcpNewRenoPlus uses a multiplicative increase in the slow 
        start phase (seen at 20 seconds for connection 3), which allows it to 
        get back to the old rate faster. 

        As for the impact on the entire network, TcpNewRenoPlus clogs up the network 
        for others. This can be seen in Connection 1 from 13 to 15 seconds, where 
        almost no packets are transmitted and congestion window remains almost 
        constant around 1 MSS. This is because Connection 2 takes up all the 
        bandwidth on the link that they share, and none is left for Connection 1.
    
\end{enumerate}

\subsection*{Appendix}

All files are as specified in the submission instructions. The fifth.cc file was 
modified to create the connection and the simulation. The script saves the congestion 
windows to the respective files, and the plotting was done using a Jupyter notebook 
and matplotlib (which has been attached, both as a notebook and as a script). 
The data files generated after the run are in the data directory.

\end{document}
